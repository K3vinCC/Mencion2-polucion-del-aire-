\documentclass[12pt,a4paper]{article}
\usepackage[utf8]{inputenc}
\usepackage[T1]{fontenc}
\usepackage[spanish]{babel}
\usepackage{geometry}
\usepackage{listings}
\usepackage{xcolor}
\usepackage{hyperref}
\usepackage{enumitem}
\usepackage{fancyhdr}
\usepackage{graphicx}
\usepackage{float}

\geometry{margin=2.5cm}
\pagestyle{fancy}
\fancyhf{}
\fancyhead[L]{\leftmark}
\fancyhead[R]{\thepage}

\lstset{
    language=JavaScript,
    basicstyle=\ttfamily\footnotesize,
    keywordstyle=\color{blue},
    stringstyle=\color{red},
    commentstyle=\color{green!60!black},
    numbers=left,
    numberstyle=\tiny,
    frame=single,
    breaklines=true,
    showstringspaces=false,
    backgroundcolor=\color{gray!10}
}

\title{Documentación de la API de Monitoreo de Calidad del Aire}
\author{Universidad Nacional}
\date{\today}

\begin{document}

\maketitle

\tableofcontents
\newpage

\section{Introducción}

Esta documentación describe cómo utilizar la API REST de Monitoreo de Calidad del Aire de la Universidad Nacional. La API permite gestionar usuarios, dispositivos IoT, lecturas ambientales y asignaciones de limpieza.

\subsection{Prerrequisitos}

\begin{itemize}
\item Python 3.8+
\item Base de datos PostgreSQL o SQLite (para desarrollo)
\item Token de autenticación JWT
\item Herramientas: curl, Postman, o cualquier cliente HTTP
\end{itemize}

\subsection{URL Base}
\begin{lstlisting}[language=]
http://localhost:5000/api
\end{lstlisting}

\section{Autenticación}

\subsection{Obtener Token de Acceso}

\textbf{Endpoint:} \texttt{POST /api/auth/login}

\textbf{Descripción:} Obtiene un token JWT para autenticar las solicitudes posteriores.

\textbf{Headers requeridos:}
\begin{itemize}
\item \texttt{Content-Type: application/json}
\end{itemize}

\textbf{Body (JSON):}
\begin{lstlisting}[language=JSON]
{
    "email": "admin@universidad.edu",
    "password": "admin123"
}
\end{lstlisting}

\textbf{Comando curl:}
\begin{lstlisting}[language=bash]
curl -X POST http://localhost:5000/api/auth/login \
  -H "Content-Type: application/json" \
  -d '{
    "email": "admin@universidad.edu",
    "password": "admin123"
  }'
\end{lstlisting}

\textbf{Respuesta exitosa (200):}
\begin{lstlisting}[language=JSON]
{
    "access_token": "eyJ0eXAiOiJKV1QiLCJhbGciOiJIUzI1NiJ9...",
    "refresh_token": "eyJ0eXAiOiJKV1QiLCJhbGciOiJIUzI1NiJ9...",
    "token_type": "Bearer",
    "expires_in": 3600,
    "user": {
        "id": 1,
        "email": "admin@universidad.edu",
        "role": "administrador"
    }
}
\end{lstlisting}

\textbf{Guardar el token para usar en las siguientes solicitudes:}
\begin{lstlisting}[language=bash]
TOKEN="eyJ0eXAiOiJKV1QiLCJhbGciOiJIUzI1NiJ9..."
\end{lstlisting}

\subsection{Verificar Token}

\textbf{Endpoint:} \texttt{GET /api/auth/verify}

\textbf{Headers requeridos:}
\begin{itemize}
\item \texttt{Authorization: Bearer \$TOKEN}
\end{itemize}

\textbf{Comando curl:}
\begin{lstlisting}[language=bash]
curl -X GET http://localhost:5000/api/auth/verify \
  -H "Authorization: Bearer $TOKEN"
\end{lstlisting}

\textbf{Respuesta exitosa (200):}
\begin{lstlisting}[language=JSON]
{
    "valid": true,
    "user": {
        "id": 1,
        "email": "admin@universidad.edu",
        "role": "administrador",
        "universidad_id": 1
    }
}
\end{lstlisting}

\section{Gestión de Usuarios}

\subsection{Obtener Lista de Usuarios}

\textbf{Endpoint:} \texttt{GET /api/users}

\textbf{Headers requeridos:}
\begin{itemize}
\item \texttt{Authorization: Bearer \$TOKEN}
\end{itemize}

\textbf{Parámetros de consulta opcionales:}
\begin{itemize}
\item \texttt{universidad\_id}: Filtrar por universidad
\item \texttt{role}: Filtrar por rol (administrador, supervisor, usuario)
\item \texttt{limit}: Número máximo de resultados (default: 50)
\end{itemize}

\textbf{Comando curl:}
\begin{lstlisting}[language=bash]
# Obtener todos los usuarios
curl -X GET http://localhost:5000/api/users \
  -H "Authorization: Bearer $TOKEN"

# Filtrar por rol
curl -X GET "http://localhost:5000/api/users?role=usuario" \
  -H "Authorization: Bearer $TOKEN"
\end{lstlisting}

\textbf{Respuesta exitosa (200):}
\begin{lstlisting}[language=JSON]
{
    "users": [
        {
            "id": 1,
            "nombre": "Administrador",
            "apellido": "Sistema",
            "email": "admin@universidad.edu",
            "rol": "administrador",
            "universidad_id": 1,
            "activo": true,
            "fecha_creacion": "2025-09-28T10:00:00Z"
        }
    ],
    "total": 1
}
\end{lstlisting}

\subsection{Obtener Usuario Específico}

\textbf{Endpoint:} \texttt{GET /api/users/\{id\}}

\textbf{Headers requeridos:}
\begin{itemize}
\item \texttt{Authorization: Bearer \$TOKEN}
\end{itemize}

\textbf{Comando curl:}
\begin{lstlisting}[language=bash]
curl -X GET http://localhost:5000/api/users/1 \
  -H "Authorization: Bearer $TOKEN"
\end{lstlisting}

\textbf{Respuesta exitosa (200):}
\begin{lstlisting}[language=JSON]
{
    "user": {
        "id": 1,
        "nombre": "Administrador",
        "apellido": "Sistema",
        "email": "admin@universidad.edu",
        "rol": "administrador",
        "telefono": "987654321",
        "universidad_id": 1,
        "activo": true,
        "fecha_creacion": "2025-09-28T10:00:00Z"
    }
}
\end{lstlisting}

\section{Gestión de Dispositivos}

\subsection{Obtener Lista de Dispositivos}

\textbf{Endpoint:} \texttt{GET /api/devices}

\textbf{Headers requeridos:}
\begin{itemize}
\item \texttt{Authorization: Bearer \$TOKEN}
\end{itemize}

\textbf{Parámetros de consulta opcionales:}
\begin{itemize}
\item \texttt{sala\_id}: Filtrar por sala
\item \texttt{tipo}: Filtrar por tipo (calidad\_aire, temperatura, humedad)
\item \texttt{activo}: Filtrar por estado (true/false)
\end{itemize}

\textbf{Comando curl:}
\begin{lstlisting}[language=bash]
# Obtener todos los dispositivos
curl -X GET http://localhost:5000/api/devices \
  -H "Authorization: Bearer $TOKEN"

# Filtrar por tipo
curl -X GET "http://localhost:5000/api/devices?tipo=calidad_aire" \
  -H "Authorization: Bearer $TOKEN"
\end{lstlisting}

\textbf{Respuesta exitosa (200):}
\begin{lstlisting}[language=JSON]
{
    "devices": [
        {
            "id": 1,
            "nombre": "Sensor A101",
            "tipo": "calidad_aire",
            "sala_id": 1,
            "token": "device_token_a101",
            "activo": true,
            "fecha_creacion": "2025-09-28T10:00:00Z"
        }
    ],
    "total": 1
}
\end{lstlisting}

\subsection{Registrar Nuevo Dispositivo}

\textbf{Endpoint:} \texttt{POST /api/devices}

\textbf{Headers requeridos:}
\begin{itemize}
\item \texttt{Authorization: Bearer \$TOKEN}
\item \texttt{Content-Type: application/json}
\end{itemize}

\textbf{Body (JSON):}
\begin{lstlisting}[language=JSON]
{
    "nombre": "Sensor Laboratorio B202",
    "tipo": "calidad_aire",
    "sala_id": 2
}
\end{lstlisting}

\textbf{Comando curl:}
\begin{lstlisting}[language=bash]
curl -X POST http://localhost:5000/api/devices \
  -H "Authorization: Bearer $TOKEN" \
  -H "Content-Type: application/json" \
  -d '{
    "nombre": "Sensor Laboratorio B202",
    "tipo": "calidad_aire",
    "sala_id": 2
  }'
\end{lstlisting}

\textbf{Respuesta exitosa (201):}
\begin{lstlisting}[language=JSON]
{
    "message": "Dispositivo registrado exitosamente",
    "device": {
        "id": 2,
        "nombre": "Sensor Laboratorio B202",
        "tipo": "calidad_aire",
        "sala_id": 2,
        "token": "generated_device_token_xyz",
        "activo": true,
        "fecha_creacion": "2025-09-28T15:30:00Z"
    }
}
\end{lstlisting}

\section{Lecturas Ambientales}

\subsection{Obtener Lecturas de Calidad del Aire}

\textbf{Endpoint:} \texttt{GET /api/readings/air-quality}

\textbf{Headers requeridos:}
\begin{itemize}
\item \texttt{Authorization: Bearer \$TOKEN}
\end{itemize}

\textbf{Parámetros de consulta opcionales:}
\begin{itemize}
\item \texttt{device\_id}: ID del dispositivo específico
\item \texttt{sala\_id}: ID de la sala
\item \texttt{fecha\_inicio}: Fecha de inicio (ISO format: YYYY-MM-DDTHH:MM:SS)
\item \texttt{fecha\_fin}: Fecha de fin (ISO format)
\item \texttt{limit}: Número máximo de lecturas (default: 100, max: 1000)
\end{itemize}

\textbf{Comandos curl:}
\begin{lstlisting}[language=bash]
# Obtener lecturas recientes de calidad del aire
curl -X GET http://localhost:5000/api/readings/air-quality \
  -H "Authorization: Bearer $TOKEN"

# Obtener lecturas de un dispositivo específico
curl -X GET "http://localhost:5000/api/readings/air-quality?device_id=1&limit=50" \
  -H "Authorization: Bearer $TOKEN"

# Obtener lecturas por rango de fechas
curl -X GET "http://localhost:5000/api/readings/air-quality?device_id=1&fecha_inicio=2025-09-28T00:00:00&fecha_fin=2025-09-28T23:59:59" \
  -H "Authorization: Bearer $TOKEN"
\end{lstlisting}

\textbf{Respuesta exitosa (200):}
\begin{lstlisting}[language=JSON]
{
    "readings": [
        {
            "id": 1,
            "dispositivo_id": 1,
            "valor_pm1": 15.2,
            "valor_pm2_5": 25.8,
            "valor_pm10": 35.4,
            "etiqueta": "ambiente",
            "fecha_lectura": "2025-09-28T14:30:00Z"
        }
    ],
    "total": 1
}
\end{lstlisting}

\subsection{Enviar Lectura desde Dispositivo IoT}

\textbf{Endpoint:} \texttt{POST /api/readings/air-quality}

\textbf{Headers requeridos:}
\begin{itemize}
\item \texttt{Authorization: Bearer \{device\_token\}}
\item \texttt{Content-Type: application/json}
\end{itemize}

\textbf{Body (JSON):}
\begin{lstlisting}[language=JSON]
{
    "pm1": 12.5,
    "pm2_5": 18.3,
    "pm10": 22.7,
    "etiqueta": "ambiente"
}
\end{lstlisting}

\textbf{Comando curl:}
\begin{lstlisting}[language=bash]
# Usar el token del dispositivo (no el JWT del usuario)
DEVICE_TOKEN="device_token_a101"

curl -X POST http://localhost:5000/api/readings/air-quality \
  -H "Authorization: Bearer $DEVICE_TOKEN" \
  -H "Content-Type: application/json" \
  -d '{
    "pm1": 12.5,
    "pm2_5": 18.3,
    "pm10": 22.7,
    "etiqueta": "ambiente"
  }'
\end{lstlisting}

\textbf{Respuesta exitosa (201):}
\begin{lstlisting}[language=JSON]
{
    "message": "Lectura registrada exitosamente",
    "reading": {
        "id": 2,
        "dispositivo_id": 1,
        "valor_pm1": 12.5,
        "valor_pm2_5": 18.3,
        "valor_pm10": 22.7,
        "etiqueta": "ambiente",
        "fecha_lectura": "2025-09-28T15:45:00Z"
    }
}
\end{lstlisting}

\subsection{Obtener Estadísticas de Calidad del Aire}

\textbf{Endpoint:} \texttt{GET /api/readings/air-quality/stats}

\textbf{Headers requeridos:}
\begin{itemize}
\item \texttt{Authorization: Bearer \$TOKEN}
\end{itemize}

\textbf{Parámetros de consulta requeridos:}
\begin{itemize}
\item \texttt{device\_id}: ID del dispositivo
\end{itemize}

\textbf{Parámetros opcionales:}
\begin{itemize}
\item \texttt{hours}: Horas para calcular estadísticas (default: 24, max: 168)
\end{itemize}

\textbf{Comando curl:}
\begin{lstlisting}[language=bash]
curl -X GET "http://localhost:5000/api/readings/air-quality/stats?device_id=1&hours=24" \
  -H "Authorization: Bearer $TOKEN"
\end{lstlisting}

\textbf{Respuesta exitosa (200):}
\begin{lstlisting}[language=JSON]
{
    "device_id": 1,
    "stats": {
        "promedio_pm1": 14.8,
        "promedio_pm2_5": 23.2,
        "promedio_pm10": 31.5,
        "max_pm2_5": 45.2,
        "min_pm2_5": 12.1,
        "lecturas_totales": 48,
        "periodo_horas": 24
    }
}
\end{lstlisting}

\subsection{Obtener Lecturas de Temperatura}

\textbf{Endpoint:} \texttt{GET /api/readings/temperature}

\textbf{Headers requeridos:}
\begin{itemize}
\item \texttt{Authorization: Bearer \$TOKEN}
\end{itemize}

\textbf{Parámetros requeridos:}
\begin{itemize}
\item \texttt{device\_id}: ID del dispositivo de temperatura
\end{itemize}

\textbf{Comando curl:}
\begin{lstlisting}[language=bash]
curl -X GET "http://localhost:5000/api/readings/temperature?device_id=3" \
  -H "Authorization: Bearer $TOKEN"
\end{lstlisting}

\textbf{Nota:} Actualmente retorna mensaje de implementación pendiente.

\subsection{Obtener Lecturas de Humedad}

\textbf{Endpoint:} \texttt{GET /api/readings/humidity}

\textbf{Headers requeridos:}
\begin{itemize}
\item \texttt{Authorization: Bearer \$TOKEN}
\end{itemize}

\textbf{Parámetros requeridos:}
\begin{itemize}
\item \texttt{device\_id}: ID del dispositivo de humedad
\end{itemize}

\textbf{Comando curl:}
\begin{lstlisting}[language=bash]
curl -X GET "http://localhost:5000/api/readings/humidity?device_id=4" \
  -H "Authorization: Bearer $TOKEN"
\end{lstlisting}

\textbf{Nota:} Actualmente retorna mensaje de implementación pendiente.

\section{Gestión de Asignaciones de Limpieza}

\subsection{Obtener Lista de Asignaciones}

\textbf{Endpoint:} \texttt{GET /api/assignments}

\textbf{Headers requeridos:}
\begin{itemize}
\item \texttt{Authorization: Bearer \$TOKEN}
\end{itemize}

\textbf{Parámetros de consulta opcionales:}
\begin{itemize}
\item \texttt{user\_id}: ID del usuario asignado
\item \texttt{sala\_id}: ID de la sala
\item \texttt{estado}: Estado de la asignación (pendiente, en\_progreso, completada, cancelada)
\item \texttt{limit}: Número máximo de resultados (default: 50)
\end{itemize}

\textbf{Comandos curl:}
\begin{lstlisting}[language=bash]
# Obtener asignaciones pendientes
curl -X GET http://localhost:5000/api/assignments \
  -H "Authorization: Bearer $TOKEN"

# Obtener asignaciones de un usuario específico
curl -X GET "http://localhost:5000/api/assignments?user_id=2&estado=pendiente" \
  -H "Authorization: Bearer $TOKEN"
\end{lstlisting}

\textbf{Respuesta exitosa (200):}
\begin{lstlisting}[language=JSON]
{
    "assignments": [
        {
            "id": 1,
            "usuario_id": 2,
            "sala_id": 1,
            "dispositivo_id": 1,
            "estado": "pendiente",
            "prioridad": "alta",
            "descripcion": "Limpieza urgente por contaminación elevada",
            "fecha_asignacion": "2025-09-28T10:00:00Z",
            "fecha_completada": null,
            "notas": null
        }
    ],
    "total": 1
}
\end{lstlisting}

\subsection{Crear Nueva Asignación}

\textbf{Endpoint:} \texttt{POST /api/assignments}

\textbf{Headers requeridos:}
\begin{itemize}
\item \texttt{Authorization: Bearer \$TOKEN}
\item \texttt{Content-Type: application/json}
\end{itemize}

\textbf{Body (JSON):}
\begin{lstlisting}[language=JSON]
{
    "usuario_id": 2,
    "sala_id": 1,
    "dispositivo_id": 1,
    "prioridad": "alta",
    "descripcion": "Limpieza urgente por contaminación elevada"
}
\end{lstlisting}

\textbf{Comando curl:}
\begin{lstlisting}[language=bash]
curl -X POST http://localhost:5000/api/assignments \
  -H "Authorization: Bearer $TOKEN" \
  -H "Content-Type: application/json" \
  -d '{
    "usuario_id": 2,
    "sala_id": 1,
    "dispositivo_id": 1,
    "prioridad": "alta",
    "descripcion": "Limpieza urgente por contaminación elevada"
  }'
\end{lstlisting}

\textbf{Respuesta exitosa (201):}
\begin{lstlisting}[language=JSON]
{
    "message": "Asignación creada exitosamente",
    "assignment": {
        "id": 2,
        "usuario_id": 2,
        "sala_id": 1,
        "dispositivo_id": 1,
        "estado": "pendiente",
        "prioridad": "alta",
        "descripcion": "Limpieza urgente por contaminación elevada",
        "fecha_asignacion": "2025-09-28T16:00:00Z"
    },
    "notification": {
        "sent": true,
        "message": "Se ha enviado una notificación al usuario asignado"
    }
}
\end{lstlisting}

\subsection{Completar Asignación}

\textbf{Endpoint:} \texttt{PUT /api/assignments/\{id\}}

\textbf{Headers requeridos:}
\begin{itemize}
\item \texttt{Authorization: Bearer \$TOKEN}
\item \texttt{Content-Type: application/json}
\end{itemize}

\textbf{Body (JSON):}
\begin{lstlisting}[language=JSON]
{
    "estado": "completada",
    "notas": "Limpieza completada satisfactoriamente. Área desinfectada."
}
\end{lstlisting}

\textbf{Comando curl:}
\begin{lstlisting}[language=bash]
curl -X PUT http://localhost:5000/api/assignments/1 \
  -H "Authorization: Bearer $TOKEN" \
  -H "Content-Type: application/json" \
  -d '{
    "estado": "completada",
    "notas": "Limpieza completada satisfactoriamente. Área desinfectada."
  }'
\end{lstlisting}

\textbf{Respuesta exitosa (200):}
\begin{lstlisting}[language=JSON]
{
    "message": "Asignación completada exitosamente",
    "assignment": {
        "id": 1,
        "estado": "completada",
        "fecha_completada": "2025-09-28T17:30:00Z",
        "notas": "Limpieza completada satisfactoriamente. Área desinfectada."
    }
}
\end{lstlisting}

\subsection{Obtener Estadísticas de Asignaciones}

\textbf{Endpoint:} \texttt{GET /api/assignments/stats}

\textbf{Headers requeridos:}
\begin{itemize}
\item \texttt{Authorization: Bearer \$TOKEN}
\end{itemize}

\textbf{Parámetros opcionales:}
\begin{itemize}
\item \texttt{days}: Días para calcular estadísticas (default: 30, max: 365)
\end{itemize}

\textbf{Comando curl:}
\begin{lstlisting}[language=bash]
curl -X GET "http://localhost:5000/api/assignments/stats?days=7" \
  -H "Authorization: Bearer $TOKEN"
\end{lstlisting}

\textbf{Respuesta exitosa (200):}
\begin{lstlisting}[language=JSON]
{
    "stats": {
        "total_asignaciones": 25,
        "completadas": 20,
        "pendientes": 3,
        "en_progreso": 2,
        "tasa_completitud": 80.0,
        "promedio_completacion_horas": 4.5
    },
    "period_days": 7
}
\end{lstlisting}

\section{Ejemplos de Flujo Completo}

\subsection{Flujo 1: Monitoreo Continuo de Calidad del Aire}

\begin{enumerate}
\item \textbf{Login como administrador:}
\begin{lstlisting}[language=bash]
curl -X POST http://localhost:5000/api/auth/login \
  -H "Content-Type: application/json" \
  -d '{"email": "admin@universidad.edu", "password": "admin123"}'
\end{lstlisting}

\item \textbf{Obtener token del dispositivo:}
\begin{lstlisting}[language=bash]
curl -X GET http://localhost:5000/api/devices \
  -H "Authorization: Bearer $TOKEN"
\end{lstlisting}

\item \textbf{Dispositivo envía lecturas periódicamente:}
\begin{lstlisting}[language=bash]
# Simular envío desde dispositivo IoT
curl -X POST http://localhost:5000/api/readings/air-quality \
  -H "Authorization: Bearer device_token_a101" \
  -H "Content-Type: application/json" \
  -d '{"pm1": 15.2, "pm2_5": 25.8, "pm10": 35.4}'
\end{lstlisting}

\item \textbf{Consultar estadísticas:}
\begin{lstlisting}[language=bash]
curl -X GET "http://localhost:5000/api/readings/air-quality/stats?device_id=1" \
  -H "Authorization: Bearer $TOKEN"
\end{lstlisting}
\end{enumerate}

\subsection{Flujo 2: Gestión de Limpieza por Contaminación}

\begin{enumerate}
\item \textbf{Detectar contaminación elevada (automático desde lecturas)}

\item \textbf{Crear asignación de limpieza:}
\begin{lstlisting}[language=bash]
curl -X POST http://localhost:5000/api/assignments \
  -H "Authorization: Bearer $TOKEN" \
  -H "Content-Type: application/json" \
  -d '{
    "usuario_id": 2,
    "sala_id": 1,
    "prioridad": "alta",
    "descripcion": "Contaminación PM2.5 > 50 µg/m³ detectada"
  }'
\end{lstlisting}

\item \textbf{Usuario asignado marca como completada:}
\begin{lstlisting}[language=bash]
curl -X PUT http://localhost:5000/api/assignments/1 \
  -H "Authorization: Bearer $USER_TOKEN" \
  -H "Content-Type: application/json" \
  -d '{"estado": "completada", "notas": "Limpieza realizada"}'
\end{lstlisting}
\end{enumerate}

\section{Códigos de Error Comunes}

\begin{itemize}
\item \textbf{400 Bad Request:} Datos inválidos en la solicitud
\item \textbf{401 Unauthorized:} Token inválido o expirado
\item \textbf{403 Forbidden:} Permisos insuficientes
\item \textbf{404 Not Found:} Recurso no encontrado
\item \textbf{500 Internal Server Error:} Error del servidor
\end{itemize}

\textbf{Ejemplo de respuesta de error:}
\begin{lstlisting}[language=JSON]
{
    "error": "Acceso denegado",
    "message": "No tienes permisos para acceder a este recurso"
}
\end{lstlisting}

\section{Datos de Prueba}

La base de datos incluye datos de prueba pre-cargados:

\begin{itemize}
\item \textbf{Universidad:} Universidad Nacional
\item \textbf{Usuarios:}
  \begin{itemize}
  \item admin@universidad.edu / admin123 (Administrador)
  \item supervisor@universidad.edu / supervisor123 (Supervisor)
  \item user@universidad.edu / user123 (Usuario regular)
  \end{itemize}
\item \textbf{Dispositivos:} 4 dispositivos de diferentes tipos
\item \textbf{Salas:} 4 salas de diferentes tipos
\end{itemize}

\section{Consideraciones de Producción}

\begin{itemize}
\item Configurar variables de entorno para producción
\item Usar HTTPS en producción
\item Implementar rate limiting avanzado
\item Configurar logs rotativos
\item Usar base de datos PostgreSQL en producción
\item Configurar CORS apropiadamente
\end{itemize}

\end{document}